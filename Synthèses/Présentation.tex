\documentclass[10pt]{beamer}

\usepackage[french]{babel}

\usepackage{preambule_diaporama_V5}
\usepackage{wasysym}
\usepackage{svg}
\usepackage{algorithm}
\usepackage{algpseudocode}
\title{Titre}
%
%% séparation colonnes
%\setlength{\columnseprule}{0.2mm}
%\setlength{\columnsep}{5mm}



\graphicspath{ {./Images/} }

\algnewcommand\algorithmicto{\textbf{to}}
\algrenewtext{For}[3]
{\algorithmicfor\ #1 $\gets$ #2 \algorithmicto\ #3 \algorithmicdo}


\begin{document}


\frame{\titlepage}

\setcounter{section}{3}

\section{Introduction}

\subsection{Enjeux}


\begin{frame}{Le labyrinthe du Big Data}
\begin{center}
\textit{Toute cette mutation en présence d'informations de plus en plus diverses et
volumineuses (Big Data) requiert des besoins de mesure nouveaux, tant en
termes de bande passante que de performances d'où la nécessité d'appareils
de mesure, de développement et d'analyse qui répondent à ces exigences
technologiques.}
\end{center}
\begin{flushright}
Un random.
\end{flushright}

\end{frame}



\subsection{Les exigences croissantes}


\begin{frame}{En cours de chargement... Veuillez patientez...}
\includegraphics[scale=0.4]{Résultat_Googles.png}
\begin{center}
	Environ \textbf{6\ 220\ 000\ 000} résultats (\textbf{0,28} secondes)
\end{center}
\end{frame}

\subsection{Problématique}
\begin{frame}{Problématique}
	\begin{center}
		\textbf{\underline{Comment identifier des signaux numériques de manière efficace ?}}
	\end{center}
\end{frame}

\section{Modélisation du problème}
\subsection{Représentation du signal numérique}
\begin{frame}{Représentation du signal numérique}
\begin{columns}
\begin{column}{0.5\textwidth}	
\begin{figure}
	\includegraphics[scale=0.3]{N_point_series.eps}
	 \caption{(Signal discret et signal continu)}
\end{figure}
\end{column}
\begin{column}{0.5\textwidth}	
	$T_e\equiv$ Période d'échantillonage
	
	$F_e\equiv$ Fréquence d'échantillonage.
	
	$N\equiv$ série de point composant.
\end{column}

\end{columns}
\end{frame}
\subsection{Représentation de la collection des musiques}
\begin{frame}{Représentation de la collection des musiques}
	\begin{itemize}
		\item On modélise la bibliothèque de musique sous forme de table de hashage
		\item On note $\mathcal{C}$ l'ensemble des clés de la table.
		\item On note $\mathcal{N}$ l'ensemble des éléments contenu dans le tableau
		\item Chaque musique est représenté par un couple $(cle,nom)$ avec $cle \in \mathcal{C}$ et $nom \in \mathcal{N}$.
		\item $nom$ correspond au nom de la musique
		\item $cle$ correspond au \textit{fingerprint} de la musique.\footnote{Le terme \textit{fingerprint} est un anglicisme voulant littéralement dire "empreinte de doigt", celle-ci étant unique pour chaque humain, on utilise ce terme pour parler d'un identifiant unique.}
		\item On note $S$ le signal numérique associé à une musique donné.
		\item On note $f:S\to C$ la fonction de hachage associant à un signal numérique donné, son \textit{fingerprint}
	\end{itemize}
\end{frame}
\section{1ère Approche - La Fast Fourier Transform}
\subsection{Définition}
\begin{frame}{Définition}
	\begin{df}{}{}
		La \textit{Fast Fourier Transform} abrégé \textit{fft} est un algorithme permettant de déterminer les coefficients de Fourier d'un signal numérique.
	\end{df}
\end{frame}
\subsection{L'algorithme de Cooley-Tukey}
\begin{frame}{Spécification}
	\begin{itemize}
		\item Il est récursif
		\item Il est basé sur le paradigme de "diviser pour régner"
		\item Sa complexité est en $O(n\log{n})$ avec $n$ le nombre de coefficient de Fourier.
		\item Entrée : Un signal sous forme de décomposition de fourrier et la liste des racine n-ième de l'unité $\mathbb{U}_n$.
		\item Sortie :  La liste des coefficients de fourrier.
	\end{itemize}
\end{frame}

\begin{frame}[t]{Principe}
\begin{itemize}
	\item Soit $x[k]$ l'amplitude du signal audio dans l'espace temporel.
\end{itemize}
\end{frame}
\begin{frame}[t]{Principe}
\begin{itemize}
	\item Soit $x[k]$ l'amplitude du signal audio dans l'espace temporel.
	\item On définit ses coefficients de Fourier $X[n]$ selon la relation suivante avec $n\in \llbracket 0, N-1\rrbracket$ :
\end{itemize}
$$X[n]=\sum^{N-1}_{k=0}x[k]e^{-2\pi i\frac{k n}{N}}$$
\end{frame}

\begin{frame}[t]{Principe}
\begin{itemize}
	\item Soit $x[k]$ l'amplitude du signal audio dans l'espace temporel.
	\item On définit ses coefficients de Fourier $X[n]$ selon la relation suivante avec $n\in \llbracket 0, N-1\rrbracket$ :
\end{itemize}
$$X[n]=\sum^{N-1}_{k=0}x[k]e^{-2\pi i\frac{k n}{N}}$$

On décompose ensuite le polynôme selon le degré paire $(k=2m)$ et impaire $(k=2m+1)$,
\end{frame}
\begin{frame}[t]{Principe}
\begin{itemize}
	\item Soit $x[k]$ l'amplitude du signal audio dans l'espace temporel.
	\item On définit ses coefficients de Fourier $X[n]$ selon la relation suivante avec $n\in \llbracket 0, N-1\rrbracket$ :
\end{itemize}
$$X[n]=\sum^{N-1}_{k=0}x[k]e^{-2\pi i\frac{k n}{N}}$$

On décompose ensuite le polynôme selon le degré paire $(k=2m)$ et impaire $(k=2m+1)$,
\begin{align*}
	X[n]&=\sum^{\lfloor \frac{N}{2}-1\rfloor}_{m=0}x[2m]e^{-2\pi i\frac{(2m)n}{N}} + \sum^{\lfloor \frac{N}{2}-1\rfloor}_{m=0}x[2m+1]e^{-2\pi i\frac{(2m+1)n}{N}}\\
	&=\underbrace{\sum^{\lfloor \frac{N}{2}-1\rfloor}_{m=0}x[2m]e^{-2\pi i\frac{(2m)n}{N}}}_{E_n}+\underbrace{e^{-2\pi i \frac{n}{N}}\sum^{\lfloor \frac{N}{2}-1\rfloor}_{m=0}x[2m+1]e^{-2\pi i\frac{(2m+1)n}{N}}}_{O_n}
\end{align*}
\end{frame}
\begin{frame}[t]{Principe}
	Ainsi,
	$$X[n]=E_n+e^{-\frac{2\pi i}{N}n}O_n$$
	$$X \left[n+\frac{N}{2}\right]=E_n-e^{-\frac{2\pi i}{N}n}O_n$$
	
	Les résultats intermédiaires $E_n$ et $O_n$ sont réutilisé. Le temps de calcul est alors réduit.
\end{frame}


\begin{frame}[t]{Principe}


%\Require Le signal sous forme de tableau : $T$
%\Require Le nombre de point sur lequel on veut faire la fft : $N$
%\Require Intervalle de saut entre les indices du tableau d’entrée $T$ :
%
%	1.	x : Il s’agit de la séquence d’entrée, c’est-à-dire le signal dont on veut calculer la FFT. Ce paramètre est généralement un tableau de nombres complexes ou réels représentant les valeurs du signal.
	%2.	N : C’est la longueur de la FFT à calculer, ou le nombre de points dans la séquence d’entrée. Cette valeur doit être une puissance de deux pour l’algorithme de Cooley-Tukey (comme 2, 4, 8, 16, etc.).
	%3.	s : Ce paramètre représente l’intervalle de saut entre les indices du tableau d’entrée x. Ce saut est utilisé pour accéder aux éléments en divisant récursivement la séquence en sous-séquences de plus en plus petites (pairs et impairs). Au début de la récursion, s est initialisé à 1 (ce qui signifie qu’on prend chaque élément de la séquence), puis il augmente en doublant à chaque étape de la récurs
	%%\State {\color{gray} // $T[:s]$ on enlève les $s$ premiers éléments tel qu'on a bien toujours $|T[:s]|\equiv 1[2]$}
	%%Isolation du morceaux, Mi-Piano- 181 point, variation dans le temps d'une amplitude.
%%%
\begin{algorithm}[H]

\caption{Algorithme de Cooley-Tukey}\label{ACT}
\begin{algorithmic}[1]
\Function{FFT}{$T,N,s$}
\If {$N=1$}
	\State \Return $T[0]$
\EndIf
\State partie\_paire $\gets$ \Call{FFT}{T,$\displaystyle\frac{N}{2}$,$2s$}
\State partie\_impaire $\gets$ \Call{FFT}{$T[:s]$,$\displaystyle\frac{N}{2}$,$2s$}
\State resultat $\gets []$
\For{i}{0}{$\frac{N}{2}$}
\State t $\gets e^{-2j*\pi\times\frac{k}{N}}\times$ partie\_ impaire $[k]$
\State resultat$[k]\gets$partie\_paire$[k] + t$
\State resultat $[k + \frac{N}{2}] \gets$ partie\_paire $ [k] - t$
\EndFor
\State \Return resultat
\EndFunction
\end{algorithmic}
\end{algorithm}
\end{frame}
\center
\begin{frame}[t]{Échantillons}
	\begin{tabular}{c c}
		Animals (Martin Garrix) & Titanium (David Guetta)\\
		\includegraphics[scale=0.2]{Album_Animals} & \includegraphics[scale=0.2]{Album_Titanium}\\
		$\approx$ 15 secondes & $\approx$ 15 secondes\\
		65535 points & 65535 points (faire différament)
	\end{tabular}
\end{frame}
\begin{frame}{Résultats}
	\includegraphics[scale=0.4]
	{Spectres_Garrix_vs_Guetta}
\end{frame}
\begin{frame}{Plusieurs problèmes}
% Plusieurs problème
% Tout d'abord pour l'utilisateur
	%1,6 Mo *
	% approx 138 T
\begin{figure}[htbp]
  \centering
  \includesvg{Fichier_Icone.svg}
  \caption{svg image}
\end{figure}
Le risque 

\end{frame}
\end{document}

