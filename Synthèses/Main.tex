q\documentclass[a4paper]{report}

%%%%%%%%%%%%%%%%%%%%%%%%%%%%%%%%%%%%%%%%%%%%%%%%%%%%%
%%%%%%%%%%%%%%%%%%%%%%%%%%%%%%%%%%%%%%%%%%%%%%%%%%%%%
%%%%%%%%%%% Chargement des paquet externe %%%%%%%%%%%
%%%%%%%%%%%%%%%%%%%%%%%%%%%%%%%%%%%%%%%%%%%%%%%%%%%%%
%%%%%%%%%%%%%%%%%%%%%%%%%%%%%%%%%%%%%%%%%%%%%%%%%%%%%

% Détection des symboles
\usepackage[utf8]{inputenc}

% Taille et configuration de la page
\usepackage{geometry}
\usepackage{fancyhdr}

% Personnalisation des titres
\usepackage{titlesec}

% Personnalisation des pieds de page et des en-têtes
\usepackage{fancyhdr}
\usepackage{etoolbox}
\usepackage{lastpage}

% \chapter est définie sur le pagestyle "plain", on redéfinie sur "fancy"
\patchcmd{\chapter}{\thispagestyle{plain}}{\thispagestyle{fancy}}{}{}

% Paquet de la matière et Raccourci commun à tout les modèles
\usepackage{./macros/macros}
\usepackage[T1]{fontenc}
\usepackage{svg}
% Graphique
\usepackage{tikz}

% Date et configuration du compilateur
\usepackage{datetime}
\usepackage{hyperref}
\usepackage{systeme}
\usepackage{amsthm}
\theoremstyle{remark}
\newtheorem{remark}{Remarque}
%%%%%%%%%%%%%%%%%%%%%%%%%%%%%%%%%%%%%%%%%%%%%%%%%%%%%
%%%%%%%%%%%%%%%%%%%%%%%%%%%%%%%%%%%%%%%%%%%%%%%%%%%%%
%%%%%%%%%%%    Configuration du modèle    %%%%%%%%%%%
%%%%%%%%%%%%%%%%%%%%%%%%%%%%%%%%%%%%%%%%%%%%%%%%%%%%%
%%%%%%%%%%%%%%%%%%%%%%%%%%%%%%%%%%%%%%%%%%%%%%%%%%%%%

%%%%% Notation Romaine
\renewcommand{\thesection}{\Roman{section}}
\renewcommand{\thesubsection}{\Roman{section}.\Roman{subsection}}



%%% Style des titres
\titleformat
{\chapter} % command
[display] % shape
{\bfseries\Large\itshape} % format
{   \rule{\textwidth}{1pt}
    \vspace{1ex}
    \centering
    Partie \thechapter
    } % label
{0ex} % sep
{
  \centering
} % before-code
[
\vspace{-0.5ex}%
\rule{\textwidth}{0.3pt}
] % after-code


% Style pour les titres de section
\titleformat{\section}
{\Huge\bfseries}
{\thesection}
{1em}
{}

\titleformat{\subsection}
{\Large\bfseries}
{\thesubsection}
{1em}
{}
% Taille et marge de la page (Format A4 marges fines)
\geometry{a4paper, margin=1.37cm}
\titlespacing*{\chapter}{0pt}{0ex plus 0ex minus 0ex}{0ex plus 0ex}

% Début du docuement
\begin{document}
%%%%%%

%%%%
% Début de la page de garde
\begin{titlepage}
  \definecolor{HR}{HTML}{2D88C4}
  % Bord gauche rouge
  \tikz[remember picture,overlay] \node[opacity=1.0,inner sep=0pt] at (current page.west){
        \begin{tikzpicture}[scale=3]
            \fill[HR] (-0.5,5) rectangle (-2,-5);
         \end{tikzpicture}
  };
    \centering
    \vspace{-40pt}

    % Photo de l'établissement
    \includegraphics[width=0.4\textwidth]{logo_CRNT.jpg}\par

    \vspace{1cm}

    {\scshape\huge TIPE \par}

    {\scshape\large Transition - Transformation - Convertion \par}

    \vspace{2.5cm}

    {\Huge\bfseries Analyse sonique par filtrage analogique et étude spectrale d'une transformé discrète de Fourier \par}

    \vspace{2cm}
    {\Large Robin PETIT, Autheur \par}
    {\Large Email: Robin\_Petit01@etu.u-bourgogne.f \par}
    \vspace{0.5cm}
    % Date
    \vfill{\large \today\par}
% Fin de la page de garde
\end{titlepage}

% Construction de la table des matières
\tableofcontents

\pagestyle{fancy}
\renewcommand{\footrulewidth}{1pt}
\fancyfoot{} 
\fancyfoot[RO,RE]{\thepage /\pageref{LastPage}}

% Initialisation de la première page
\newpage

\chapter{Mécanique}
\section{Acquisition du son}
\subsection{Propagation}
\subsubsection{Caractère ondulatoire}
Le son peut être représenter sous la forme d'une onde\textbf{ {\color{bleu}mécanique}  {\color{violet}longitudinale} {\color{cyan}plane} {\color{vert}progressive}}.
\begin{remark}
	Ce TIPE pars du principe que le milieu est toujours isotrope.
\end{remark}


\begin{itemize}
\item[$\bullet$] 	\textbf{\color{bleu}mécanique} : Perturbation dans un milieu matériel.
\item[$\bullet$] 	\textbf{\color{violet}longitudinale} : Perturbation dans la direction de propagation.
\item[$\bullet$] 	\textbf{\color{cyan}plane} : Front d'onde contenu dans un plan.
\item[$\bullet$] 	\textbf{\color{vert}progressive} : Transport d'énergie dans toute direction accessible.
\end{itemize}

\subsubsection{Modèle d'onde plane harmonique}
\lformule{\center $\psi(x,t)=A\cos{\omega t-kx+\varphi}$}{
A : Amplitude de Vibration

$\omega$ : pulsation

$k$ : nombre d'onde
}

\subsection{Réception}
\subsubsection{Microphone électrostatique}

\newpage
% Fin du document
\end{document}